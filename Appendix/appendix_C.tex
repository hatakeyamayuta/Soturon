\chapter{本論文内容に関係する研究アイデア}
\lhead[付録C 本論文内容に関係する研究アイデア]{}
\setcounter{page}{1}
\renewcommand{\thepage}{C--\arabic{page}}

\thispagestyle{empty}

\newpage
%%%%%%%%%%%%%%%%%%%%%%%%%%%%%%%%%%%%%%%%%%%%%%%%%%%%%%%%%%%%%%%%%%%%%%%%%%%%%%%

\section{はじめに}



\newpage

\section{屈折画像1枚と普通の画像1枚でスケール復元可能なStructure from Motion}



\newpage

%\section{対応点検出時のRANSACについて}
%一般的なStructure from Motionにおいて,画像から対応点を検出する際,
%3次元復元に有効な対応点を見つける手法として,RANSAC(RANdom SAmple Consensus)が利用されることが多い.
%これは,検出した対応点の中から数点を選び,それらを用いて基礎行列を算出し,この基礎行列が正しいか,他の点を用いて検証するという手法である.
%しかし屈折画像を用いている提案手法において,この一般的なRANSACを用いることはできない.なぜならば,一般的なStructure from Motionの基礎方程式は提案手法において,
%\begin{equation}
%\left\{({\mathbf t}+{\mathbf R}^{-1}{\mathbf d'}-{\mathbf d}) \times {\mathbf R}^{-1}{\mathbf r'}\right\}^{\mathrm{T}}{\mathbf r}
%= 0,
%\end{equation}
%\newpage


%%%%%%%%%%%%%%%%%%%%%%%%%%%%%%%%%%%%%%%%%%%%%%%%%%%%%%%%%%%%%%%%%%%%%%%%%%%%%%%
%%% Local Variables:
%%% mode: katex
%%% TeX-master: "../thesis"
%%% End:
