\chapter*{参考文献}
%\addcontentsline{toc}{chapter}{参考文献}
\lhead[参考文献]{}
%\renewcommand{\refname}{引用文献}
\thispagestyle{empty}

\newpage

%==============================================================================
%和文文献
%==============================================================================
\subsection*{\textmc{<和文文献>}}
\begin{mythebibliography}{}

\bibitem[日本建設組合連合 2016]{日本建設組合連合2016}
\leavevmode\\
\newblock ``日本経済と建設業'',
\newblock http://www.nikkenren.com/publication/pdf/handbook/2016/201602.pdf, 2016, 閲覧日~2019.12.24.
\\

\bibitem[建設経済研究所 2017]{建設経済研究所2017}
\leavevmode \\建設経済研究所:
\newblock ``建築産業の現状と課題'',
\newblock 建設経済レポート, Vol.~69, pp.~182--213, 2017.
\\
\bibitem[国土交通省 2015]{国土交通省2015}
\leavevmode \\国土交通省:
\newblock ``建設業'',
\newblock http://www.mlit.go.jp/common/001191669.pdf, 2015, 閲覧日~2019.12.24.
\\
\bibitem[土井下 2010]{土井下2010}
\leavevmode \\土井下~健治, 村本~英一, 神田~俊彦:
\newblock ``建設機械へのICT応用'',
\newblock Komatsu Technical Report, Vol.~56, No.~163, 2010.
\\
\newpage

\bibitem[中原智治 2001]{中原智治2001}
\leavevmode \\中原智治,顧海松,荒木秀和,藤井裕之:
\newblock ``3次元認識によるビンピッキングシステムの実用化'',
\newblock システム制御情報学会論文誌, Vol.~14, No.~4, pp.~226--232, 2001.
\\
\bibitem[林 2008]{林2008}
\leavevmode \\林俊寛,曽根原光治,井之上智洋,島輝行.河野幸弘:
\newblock ``三次元物体認識技術を応用したバラ積みピッキングシステムの開発'',
\newblock IHI技報,Vol.~48, No.~1, pp.~7--11, 2008.
\\
\bibitem[西卓郎 2014]{西卓郎2014}
\leavevmode \\西卓郎, 吉見隆,高瀬竜一,原田研介,永田和之,新良貴陽平,河井良浩:
\newblock ``ビンピッキングのためのRGB-Dカメラを用いた三次元位置姿勢推定,および把持可能性を考慮したスコアリング手法'',
\newblock 第157回GCAD・第194回CVIM合同研究発表会,pp.~1--6, 2014
\\

%\bibitem[金出 1993]{金出1993}
%\leavevmode \\金出武雄, コンラッド ポールマン, 森田俊彦:
%\newblock ``因子分解法による物体形状とカメラ運動の復元'',
%\newblock 電子情報通信学会論文誌, Vol.~J76-D-2, No.~8, pp.~1497--1505, 1993.
%\\
%

%
%\bibitem[久米 2014]{久米2014}
%\leavevmode \\久米大将, 藤井浩光, 山下淳, 淺間一:
%\newblock ``全方位カメラを用いたスケール復元が可能な水中Structure from Motion'',
%\newblock 2014年度精密工学会春季大会学術講演会講演論文集, \mbox{pp.~981--982}, 2014.
%\\
%
%\bibitem[松井 2010]{松井2010}
%\leavevmode \\松井建樹, 山下淳, 金子透:
%\newblock ``全方位レーザ・全方位カメラからなるレンジファインダの自己位置推定と配管の3次元モデル生成'',
%\newblock 電気学会論文誌C, Vol.~130, No.~9, pp.~1504--1512, 2010.
%\\
%
%\bibitem[松久 2008]{松久2008}
%\leavevmode \\松久亮太, 川崎洋, 小野晋太郎, 阪野貴彦, 池内克史:
%\newblock ``因子分解法とバンドル調整による全方位画像列からの形状および位置姿勢の同時推定手法'',
%\newblock 画像の認識・理解シンポジウム(MIRU2008), pp.~1610--1617, 2008.
%\\
%

%
%\bibitem[東京電力株式会社 HP]{東京電力株式会社HP}
%\leavevmode \\東京電力株式会社:
%\newblock 福島第一原発原子炉建屋周辺の様子(画像),\\
%\newblock http://photo.tepco.co.jp/cat2/01-j.html, 2014.
%\\



\newpage
%==============================================================================
%英文文献
%==============================================================================
\subsection*{\textmc{\hspace{-1zw}<英文文献>}}

\bibitem[Mousavian 2017]{2017} 
\leavevmode \\ D. Anguelov, J. Flynn and J. Kosecka:
\newblock ``3DBounding Box Estimation Using Deep Learning and Geometry'',
\newblock  The IEEE Conference on Computer Vision and Pattern Recognition, pp.~7074--7082, 2017.
\\
            
\bibitem[Zhang 2017]{Zhang2017}
\leavevmode \\ X. Zhang, W. Xu, C. Dong and J. M. Dolan:
\newblock ``Efficient L-Shape Fitting for Vehicle Detection Using Laser Scanners'',
\newblock The IEEE Intelligent Vehicles Symp, 2017.
\\
\bibitem[Chen 2017]{Chen2017}
\leavevmode \\X. Chen, H. Ma, J. Wan, B. Li, T. Xia:
\newblock ``Pointpillars: Fast encoders for object detection from point clouds'',
\newblock The IEEE Conference on Computer Vision and Pattern Recognition, pp. 1907-1915, 2017.
\\
\bibitem[Lang 2019]{Lang2019}
\leavevmode \\A. H. Lang, S. Vora, H. Caesar, L. Zhou, J. Yang, and O. Beijbom:
\newblock ``Pointpillars: Fast encoders for object detection from point clouds'',
\newblock The IEEE Conference on Computer Vision and Pattern Recognition, 2019.
\\


\bibitem[Geiger 2012]{Geiger2012} 
\leavevmode \\L. A. Geiger and R. Urtasun:
\newblock ``Are we ready for autonomousdriving? the KITTI vision benchmark suite'',
\newblock The IEEE Conference on Computer Vision and Pattern Recognition, pp.~3354--3361, 2012.
\\

\bibitem[Garrido 2015]{Garrido2015} 
\leavevmode \\S. Garrido-Jurado, R. Muñoz-Salinas, F. J. Madrid-Cuevas and J. Medina-Carnicer:
\newblock ``Generation of Fiducial Marker Dictionaries Using Mixed Integer Linear Programming'',
\newblock Pattern Recognition, No.~51, pp.~481--491, 2015.
\\

\bibitem[Sundermeyer 2018]{Sundermeyer2018}
\leavevmode \\M. Sundermeyer, Z. C. Marton, M. Durner, M. Brucker and R. Triebel:
\newblock ``Implicit 3D Orientation Learning for 6D Object Detection from RGB Images'',
\newblock The European Conference on Computer Vision, pp.~699--715, 2018.
\\
\bibitem[Tremblay 2018]{Tremblay2018}
\leavevmode \\J. Tremblay, T. To, B. Sundaralingam, Y. Xiang, D. Fox,and S. Birchfield:
\newblock ``Deep object pose estimation for seman-tic robotic grasping of household objects'',
\newblock Arxiv preprint arxiv:1809, 10790, 2018.
\\
\bibitem[Rusu 2009]{Rusu2009}
\leavevmode \\R.B.Rusu, N.Blodow and M.Beetz:
\newblock ``Fast Point Fea-ture Histograms (FPFH) for 3D registration'',
\newblock Interna-tional Conference, pp.~3212--3217, 2009.
\\

\bibitem[Fischler 1981]{Fischler1981}
\leavevmode \\Martin A. Fischler and Robert C. Bolles:
\newblock ``Random Sample Consensus: a Paradigm for Model Fitting with Applications to Image Analysis and Automated Cartography'',
\newblock Communications of the ACM, Vol.~24, No.~6, pp.~381--395, 1981.
\\

\bibitem[Chetverikov 2008]{Chetverikov2008} 
\leavevmode \\D. Chetverikov, D. Svirko, D. Stepanov and P. Krsek:
\newblock ``Aligning Point Cloud Viewsusing Persistent Feature Histograms'',
\newblock Proceedings of the  IROS, pp.~3384--3391, 2008.
\\

\bibitem[Chetverikov 2002]{Chetverikov2002}
\leavevmode \\D. Chetverikov, D. Svirko, D. Stepanov and P. Krsek,:
\newblock ``The Trimmed Iterative Closest Point Algorithm'',
\newblock  International  Conference  on Pattern Recognition, pp.~545--548, 2002.
\\



%\bibitem[Agarwal 2009]{Agarwal2009} 
%\leavevmode \\Sameer Agarwal, Noah Snavely, Ian Simon, Steven M. Seitz and Richard Szeliski:
%\newblock ``Building Rome in a Day'',
%\newblock Proceedings of the 2009 IEEE International Conference on Computer Vision, pp.~72--79, 2009.
%\\

%
%\bibitem[Chang 2011]{Chang2011}
%\leavevmode \\Yao-Jen Chang and Tsuhan Chen:
%\newblock ``Multi-View 3D Reconstruction for Scenes under the Refractive Plane with Known Vertical Direction'',
%\newblock Proceedings of the 2011 IEEE International Conference on Computer Vision, pp.~351--358, 2011.
%\\

%
%\newpage
%


%
%
%\bibitem[Kukelova 2011]{Kukelova2011}
%\leavevmode \\Zuzana Kukelova, Martin Bujnak and Tomas Pajdla:
%\newblock ``Closed-form solutions to the minimal absolute pose problems with known vertical direction'',
%\newblock Proceedings of the Asian Conference on Computer Vision, pp.~216--229, 2010.
%\\
%
%\bibitem[Lee 1999]{Lee1999} 
%\leavevmode \\Doo Hyun Lee, In So Kweon and Roberto Cipolla:
%\newblock ``A Biprism--Stereo Camera System'',
%\newblock Proceedings of the 1991 IEEE Computer Society Conference on Computer Vision and Pattern Recognition, Vol.~1, pp.~82--87, 1999.
%\\
%

%
%\bibitem[Snavely 2006]{Snavely2006}
%\leavevmode \\Noah Snavely, Steven M. Seitz and Richard Szeliski:
%\newblock ``Photo Tourism: Exploring Photo Collections in 3D'',
%\newblock ACM Transactions on Graphics, Vol.~25, pp. 835--846 , 2006.
%\\
%
%\bibitem[Tomasi 1992]{Tomasi1992}
%\leavevmode \\Carlo Tomasi and Takeo Kanade:
%\newblock ``Shape and Motion from Image Streams under Orthography: a Factorization Method'',
%\newblock International Journal of Computer Vision, Vol.~9, No.~2, pp.~137--154, 1992.
%\\
%
%\bibitem[Treibitz 2011]{Treibitz2011}
%\leavevmode \\Tali Treibitz, Yoav Y. Schechner and Hanumant Singh:
%\newblock ``Flat Refractive Geometry'',
%\newblock IEEE Transactions on Pattern Analysis and Machine Intelligence, Vol.~34, No.~1, pp.~51--65, 2011.
%\\
%

%\bibitem[Yamashita 2010]{Yamashita2010} 
%\leavevmode \\Atsushi Yamashita, Yudai Shirane and Toru Kaneko:
%\newblock ``Monocular Underwater Stereo -- 3D Measurement Using Difference of Appearance Depending on Optical Paths --'',
%\newblock Proceedings of the 2010 IEEE/RSJ International Conference on Intelligent Robots and Systems, pp.~3652--3657, 2010.
%\\
%
%\newpage
%
%\bibitem[Yamashita 2011]{Yamashita2011}
%\leavevmode \\Atsushi Yamashita, Kenki Matsui, Ryosuke Kawanishi, Toru Kaneko, Taro Murakami, Hayato Omori, Taro Nakamura and Hajime Asama:
%\newblock ``Self-Localization and 3-D Model Construction of Pipe by Earthworm Robot Equipped with Omni-Directional Rangefinder'',
%\newblock Proceedings of the 2011 IEEE International Conference on Robotics and Biomimetics, \mbox{pp.~1017--1023}, 2011.
%\\
%






%scale problem
%・Accurate Scale Factor Estimation in 3D Reconstruction
%\bibitem{4} Manolis Lourakis and Xenophon Zabulis: ``Accurate Scale Factor Estimation
%in 3D Reconstruction'',
%{\it Proceedings of the International Conference on Computer Analysis of Images and Patterns}, 
%Vol.1, pp.498--506, 2013.


%Plate calib
%・A Theory of Multi-Layer Flat Refractive Geometry
%Amit Agrawal, Srikumar Ramalingam, Yuichi Taguchi, Visesh Chari:
%Proceedings of the 2012 IEEE Conference on Computer Vision and Pattern Recognition, pp.~2316--2323, 2006.


%・Flat Refractive Geometry
%\bibitem{5} Tali Treibitz, Yoav Y. Schechner and Hanumant Singh: ``Flat Refractive
%Geometry'', 
%IEEE Transactions on Pattern Analysis and Machine Intelligence, Vol.~34, No.~1, pp.~51--65, 2011.


%with IMU 
%・Multi-View 3D Reconstruction for Scenes under the Refractive Plane with Known Vertical Direction
%\bibitem{13} Yao-Jen Chang and Tsuhan Chen: ``Multi-View 3D Reconstruction for Scenes under the Refractive Plane with Known Vertical Direction'',
%{\it Proceedings of the 2011 IEEE International Conference on Computer Vision}, 
%pp.351--358, 2011.

%・Closed-form solutions to the minimal absolute pose problems with known vertical direction
%Zuzana Kukelova, Martin Bujnak, Tomas Pajdla:
%Computer Vision -- ACCV 2010, pp.~216--229, 2011.

%with rangefinder
%・Self-Localization and 3-D Model Construction of Pipe by Earthworm Robot Equipped with Omni-Directional Rangefinder
%Atsushi Yamashita, Kenki Matsui, Ryosuke Kawanishi, Toru Kaneko, Taro Murakami, Hayato Omori, Taro Nakamura and Hajime Asama: "Self-Localization and 3-D Model Construction of Pipe by Earthworm Robot Equipped with Omni-Directional Rangefinder", Proceedings of the 2011 IEEE International Conference on Robotics and Biomimetics (ROBIO2011), pp.1017-1023, 2011.


%explain SfM 
%・Shape and Motion from Image Streams under Orthography: a Factorization Method 
%International Journal of Computer Vision, Vol.~9, No.~2, pp.~137--154, 1992 

%Bundle Adjustment -- A Modern Synthesis
%Bill Triggs, Philip F. McLauchlan, Richard I. Hartley, and Andrew W. Fitzgibbon:
%Vision Algorithms: Theory and Practice, pp.~298--372, 2000.

%absolute pose
%Absolute Pose for Cameras Under Flat Refractive Interfaces


\end{mythebibliography}