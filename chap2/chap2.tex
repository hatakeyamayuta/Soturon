\chapter{提案手法}
\thispagestyle{empty}
\label{chap2}
\minitoc

\newpage
%%%%%%%%%%%%%%%%%%%%%%%%%%%%%%%%%%%%%%%%%%%%%%%%%%%%%%%%%%%%%%%%%%%%%%%%%%%%%%%
%==============================================================================
%はじめに
%==============================================================================
\section{はじめに}
本章では,画像による3次元物体検出と点群位置合わせによるダンプトラックの位置姿勢推定をするための手法について述べる.
\par
2.2 節では,ダンプトラックの位置姿勢を推定するために画像による3次元物体と点群位置合わせを統合したダンプトラックの位置姿勢推定のアプローチについて述べる.
\par
2.3 節では,画像による3次元物体検出の手法について述べる.
\par
2.4 節では,点群位置合わせによるダンプトラックの位置姿勢推定について述べる.
\newpage

\section{位置姿勢推定のアプローチ}
\subsection{画像による3次元物体検出と点群位置合わせによる位置姿勢の概要}
第 1 章で述べたように本研究ではダンプトラックの位置姿勢を計測するために画像による3次元物体検出と点群位置合わせを統合した位置姿勢推定の手法を用いる.
その概要について説明する.
\par
事前に点群位置合わせの基準となるダンプトラックの3次元モデルを作成を行う.バックホウに搭載した複数台のRGB-Dセンサから土砂積み込み作業範囲に
設置したダンプトラックを計測し3次元点群を取得する.取得した3次元点群は地面情報やノイズを含むためダンプトラックの点群を抽出することで3次元モデルを作成する.
\par
次に位置姿勢推定の概要について説明する.
ダンプトラックは土砂積み込み作業範囲外からバックホウに向かって進入すると仮定する.遠方からバックホウに向かって進入するダンプトラックを
バックホウに搭載したRGB-Dセンサから撮影し,画像による3次元物体検出により大まか位置姿勢を計測する.また,推定値が土砂積み込み作業範囲外であれば範囲内に進入するまで3次元物体検出を行い,
範囲内であれば,推定値を初期値とした,点群位置合わせにより位置姿勢推定を行う.
点群位置合わせは基準モデルと計測データが必要だが,基準モデルには事前に作成したダンプトラックの3次元モデル,計測データにはバックホウに搭載したRGB-Dセンサから計測した
3次元点群を用いる.
また,3次元物体検出の際,ダンプトラックが映っているのにかかわらず推定に失敗する場合ある.そのため,検出に失敗した場合は直前のフレームを参照し,ダンプトラックの位置が土砂積み込み
作業範囲内であれば3次元特徴量マッチングを初期値とした点群位置合わせにより位置姿勢推定を行う.
\newpage
\subsection{位置姿勢推定システムの概要}
本節では,画像による3次元物体検出や点群位置姿勢に必要となる画像や点群を計測する方法を説明する.

\newpage
\section{画像による3次元物体検出}
\subsection{深層学習による3次元物体検出}
本項では深層学習による3次元物体検出について述べる.本研究では単一の画像から求めたCNN特徴量から対象物の3次元座標と姿勢,
画像内の領域を推定できる3D Bounding Box Estimation Using Deep Learningand Geometry\cite{2017}を用いる.

\newpage

\subsection{データセットの作成}
本項では深層学習による3次元物体検出によりダンプトラックを認識するために必要な学習用データセットの作成方法について述べる.
2.3.1 節で述べたように,深層学習による3次元物体検出は一般車両の自動運転を目的としたものであるため,ダンプトラックのような建設機械は
データ数が少ないため認識精度は低い.そのため本研究では,認識精度向上させるためにダンプトラックの模型を用いた学習用のデータセット作成を行う.
データセットを作成するためにはダンプトラックの画像,位置姿勢,寸法と対象物体の画像座標が必要となる.そのため,図に示すように,
回転台の上にダンプトラックを設置することで回転台から姿勢を計測し,RGB-Dカメラにより画像と位置を獲得する.その後,撮影した画像からダンプトラックの画像座標を
ラベル付することでデータセットの作成を行う.
また,データセットを拡張するために背景をクロマキー合成する.

\newpage

\section{点群位置合わせによる位置姿勢推定}
\subsection{ダンプトラックの点群抽出}
本項では距離センサから計測した点群から点群位置合わせの基準となるダンプトラックの点群を抽出する方法を述べる.

\newpage

\subsection{点群位置合わせ}
本項では点群位置合わせによるダンプトラックの位置姿勢の推定手法ついて述べる.
点群位置合わせにはICP(Iterative ClosestPoint)を用いる.


\newpage

\subsection{3次元特徴量マッチング}
本項では点群位置合わせの初期値となる3次元特徴量マッチングについて述べる.

\newpage

\section{おわりに}

\newpage