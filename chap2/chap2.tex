\chapter{提案手法}
\thispagestyle{empty}
\label{chap2}
\minitoc

\newpage
%%%%%%%%%%%%%%%%%%%%%%%%%%%%%%%%%%%%%%%%%%%%%%%%%%%%%%%%%%%%%%%%%%%%%%%%%%%%%%%
%==============================================================================
%はじめに
%==============================================================================
\section{はじめに}
本章では,画像による3次元物体検出と点群位置合わせによるダンプトラックの位置姿勢推定をするための手法について述べる.
\par
2.2 節では,ダンプトラックの位置姿勢を推定するために画像による3次元物体と点群位置合わせを統合したダンプトラックの位置姿勢推定のアプローチについて述べる.
\par
2.3 節では,画像による3次元物体検出の手法について述べる.
\par
2.4 節では,点群位置合わせによるダンプトラックの位置姿勢推定について述べる.
\newpage

\section{位置姿勢推定のアプローチ}
\subsection{画像による3次元物体検出と点群位置合わせによる位置姿勢の概要}
第 1 章で述べたように本研究ではダンプトラックの位置姿勢を計測するために画像による3次元物体検出と点群位置合わせを統合した位置姿勢推定の手法を用いる.
その概要について説明する.
\par
事前に点群位置合わせの基準となるダンプトラックの3次元モデルを作成を行う.バックホウに搭載した複数台のRGB-Dセンサから土砂積み込み作業範囲に
設置したダンプトラックを計測し3次元点群を取得する.取得した3次元点群は地面情報やノイズを含むためダンプトラックの点群を抽出することで3次元モデルを作成する.
\par
次に位置姿勢推定の概要について説明する.
ダンプトラックは土砂積み込み作業範囲外からバックホウに向かって進入すると仮定し,バックホウに向かって進入するダンプトラックを
バックホウに搭載したRGB-Dセンサから撮影し,画像による3次元物体検出を行う.推定値が土砂積み込み作業範囲外であれば3次元物体検出を続けて行い,
範囲内であれば,推定値を初期値として入力を行い,計測した3次元点群と事前に作成した3次元モデルとの点群位置合わせにより位置姿勢推定を行う.
また,3次元物体検出はダンプトラックが映っているのにかかわらず未検出となる場合ある.未検出の際は前フレームを参照し,ダンプトラックの位置が土砂積み込み
作業範囲内であれば3次元特徴量マッチングを初期値とした点群位置合わせにより位置姿勢推定を行う.
\newpage
\subsection{位置姿勢推定システムの概要}
本節では,画像による3次元物体検出や点群位置姿勢に必要となる画像や点群を計測する方法を説明する.

\newpage
\section{画像による3次元物体検出}
\subsection{深層学習による3次元物体検出}
\newpage
\subsection{データセットの作成}

\section{点群位置合わせによる位置姿勢推定}
\subsection{ダンプトラックの点群抽出}
\subsection{点群位置合わせ}
\subsection{3次元特徴量マッチング}
\section{おわりに}
