\chapter{序論}
\thispagestyle{empty}
\label{Chap1}
\minitoc

\newpage
%%%%%%%%%%%%%%%%%%%%%%%%%%%%%%%%%%%%%%%%%%%%%%%%%%%%%%%%%%%%%%%%%%%%%%%%%%%%%%%

%==============================================================================
%背景
%==============================================================================
\section{背景}
\label{Background}
建設業は,道路、河川などの社会資本や産業施設,公共施設の整備・維持管理を行い,国内総生産及び就業者数の約10%を占める基幹産業の一つである.\cite{日本建設組合連合2016}
\par 2010 年に発生した東日本大震災や各地の豪雨災害時での復興活動などで,建設業の重要性が再認識されているが,
しかし,近年の建設業界では,技能労働者の高齢化や就業者の減少により熟練オペレーター不足が問題となってい
る.また,国土交通省の「建築産業の現状と課題」\cite{建設経済研究所2017}によると,2015 年の技能労働者数は330 万人であり,10 年後
の 2025 年は 286 万人と減少すると試算されている.今後,深刻な人材不足の危機に陥ると予想されており,人材不
足を補う為,建設現場における作業の自動化は重要な課題である.
現場における作業で自動化の要求の高い作業の一つが,バックホウとダンプトラックの連携による土砂積み込み作業の自動化である.土砂の積み込みの際には,ダ
ンプトラックは運転手により積み込み位置まで移動されるが,バックホウによる積み込み作業を自動化するために
は,バックホウに対するダンプトラックの相対的な位置姿勢を正しく獲得する必要がある.
一般の作業現場では,GNSSやTotal Stationによって位置姿勢の計測\cite{土井下2010}が行われているが,通信基地局などの環境整備が必要であり,
設備コストが大きいことが課題である.そのため,環境に大がかりな設備を要さずに位置姿勢を計測する手法が期待されている.
%%%%%
\begin{figure}[b]
 \begin{center}
 \includegraphics[width=1.0\columnwidth]{./chap1/fig/GNSS.png}
 \caption{GNSSやTSによる位置情報の把握}
 \label{fig:GNSS}
 \end{center}
 %\vspace{-5mm}
\end{figure}
%%%%%

\newpage
\section{従来研究}

\section{研究の目的}
1.1 節で述べたように,
\par
そこで,本研究では
    \begin{screen}
        \begin{center}
        計測範囲の拡張と距離センサの計測範囲外でも計測可能なダンプトラックの位置姿勢推定手法の提案
        \end{center}
    \end{screen}
を目的とする
\section{本論文の構成}
%%%%%%%%%%%%%%%%%%%%%%%%%%%%%%%%%%%%%%%%%%%%%%%%%%%%%%%%%%%%%%%%%%%%%%%%%%%%%%%
%%% Local Variables:
%%% mode: katex
%%% TeX-master: "../thesis"
%%% End: